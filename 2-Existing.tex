\section{Existing EUV wave detection algorithms}\label{sec:existing}

Automated feature detection algorithms have an advantage over human detections of features because
they generate repeatable results for the same input data, i.e. they enable reproducability. In addition, their ability to examine large quantities of data faster than human analysis is invaluable in the SDO era, which produces ~1 TB of solar data each day. The solar physics community already makes use of the Computer Aided CME Tracking (CACTus: \cite{2004A&A...425.1097R}) and Solar Eruptive Event
Detection System (SEEDS: \cite{2008SoPh..248..485O}) CME catalogs,
both of which are generated from automated feature detection
algorithms. 

When relying on such automated procedures, it is scientifically valuable to have multiple independently designed and developed automatic feature detection algorithms for the same type of
feature as they allow for cross-checking and verification of results.
By comparing the final results, the operation of each feature
detection algorithm can be better understood. For example, \cite{2013ApJ...768..162P} shows that
eruption rates from both the CACTus and SEEDS CME catalogs are
systematically higher in 2003-2012 compared to 1997-2002, consistent
with the weakness of the late cycle 23 polar fields.  Use of
detections from two independently developed automated CME detection
algorithms greatly strengthens this result.

In terms of EUV waves, there are at least two automated EUV wave detection methods currently
published, the Novel EUV wave Machine Observing algorithm, described
by \cite{2005SoPh..228..265P} (see also
\cite{2012SoPh..276..479P}) and the Coronal Pulse Identification and
Tracking Algorithm described in \cite{2014SoPh..289.3279L}. NEMO was originally designed for analysis of SOHO/EIT data, but has since been modified to analyze STEREO/EUVI images. The original NEMO algorithm \cite{2005SoPh..228..265P} consists of three
components. These are 1) source event detection, 2) recognition of
eruptive dimmings, 3) detection and analysis of EUV waves. The event
detection component is based on the higher-order moments of running
difference (RD) images. A RD image is simply the difference between
two consecutive images. A sharp change in the skewness or kurtosis of
the distribution of RD image values is a reliable signature that an
impulsive event, such as a flare or an EUV wave, has been observed in
that image. Once a source detection has occurred, the second and third
components of NEMO are triggered to identify eruptive dimmings and EUV
waves respectively.  NEMO assumes that EUV waves propagate circularly
from the originating event. The wave front is found by integrating RD
images in nested annuli centered on the originating event.  Plotting
these integrals as a function of radial distance from the originating
event shows a leading intensity enhancement which is identified as the
EUV wave.  Results from NEMO are available at
\url{http://sidc.be/nemo/}; however, the implementation ceased
operations in 2010 and so there are no new EUV detections being
provided to the community.  
\cite{2012SoPh..276..479P} is concerned with advances to original NEMO
algorithm with respect to source event detection and eruptive
dimmings, and does not explicitly tackle EUV waves.

The second algorithm that has been developed is CorPITA
\citep{2014SoPh..289.3279L}. CorPITA uses percentage base-difference
(PBD) images as the foundation for detection (Fig. 2, top row).  A PBD
image is formed by taking the difference between a selected base image
and the current image, and then scaling that difference by the base
image, multiplied by 100.  CorPITA is triggered by the occurrence of a
flare.  In CorPITA PBD images, the base image is taken two minutes
before the flare start time. The flare position is used as the origin
of the EUV wave; great circles intersecting this origin are analyzed
to identify whether an EUV wave is present. The intensity profile
along the great circle is fitted for each time-step with a
multi-Gaussian function, based on the observation of
\cite{2006ApJ...645..757W} that cross-sections of EUV wave events have
this approximate form. This assumption allows the wave to be
characterized in terms of its position, velocity and width. Events and
data products generated by CorPITA are intended to be made available
within the HEK \citep{hek2012, 2012SoPh..275...79M}.  

In this context, AWARE provides a new, alternative approach for the detection and characterization of EUV waves. Unlike the above algorithms, AWARE was designed from the beginning to make use of the high-resolution data provided by SDO/AIA. In the following Sections, we describe in detail the imaging processing steps in AWARE, and demonstrate its application to solar data.
