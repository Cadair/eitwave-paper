\section{Introduction}\label{sec:Intro}

Extreme ultraviolet (EUV) waves are large-scale propagating
disturbances observed in the solar corona, frequently associated with
coronal mass ejections and flares.  Since their discovery
\citep{1997SoPh..175..571M, 1998GeoRL..25.2465T, 1999ApJ...517L.151T}
over two hundred papers discussing their properties, causes and
physics have been published.  However, their fundamental nature is
still not understood. In general, studies of this phenomenon can be
assigned to at least one of these broad, non-exclusive categories: the
physical nature and appearance of EUV waves, investigation of
correlated phenomena, such as CMEs, flares, dimmings, and filament
activity, probing the origin or driver of EUV waves, understanding the
interaction with and impact on existing coronal features.

In each of these categories, there have been major breakthroughs in
the last several years, primarily due to the availability of
high-cadence, multi-wavelength, multi-viewpoint observations from SDO,
STEREO, Hinode, and other sources (for comprehensive reviews of recent
results see \cite{2011SSRv..158..365G}; \cite{2011JASTP..73.1096Z};
\cite{2011A&A...532A.151W}, \cite{2012SoPh..281..187P}).  Careful
analysis has yielded a much improved understanding of the EUV wave
phenomenon (e.g., Fig. 1), but it is clear that many outstanding
questions remain.

At a fundamental level, the physical nature of EUV waves is not
completely understood. Some studies present evidence supporting a
magnetohydrodynamic (MHD) wave interpretation
\citep{1998GeoRL..25.2465T, 1999ApJ...517L.151T,2000ApJ...543L..89W,
  2001JGR...10625089W, 2002ApJ...574..440O, 2010ApJ...713.1008S},
others argue for what \cite{2012SoPh..281..187P} call a “pseudo-wave”
due to either the evolving manifestations of a CME
\citep{1999SoPh..190..107D, 2000ApJ...545..512D, 2008SoPh..247..123D,
  2011ApJ...738..167S} or transient localized brightenings
\citep{2007AN....328..760A, 2007ApJ...656L.101A,}.  Some authors have
found evidence indicating that the complex brightenings associated
with EUV waves can be due to a combination of both MHD waves and
pseudo-waves \citep{2002ApJ...572L..99C, 2005ApJ...622.1202C,
  2004A&A...427..705Z, 2009ApJ...705..587C}.  It is clear from the
literature that the physical conditions that lead to the broad range
of observed wave propagation speeds \citep{2011A&A...532A.151W} and
amplitudes are poorly understood.

EUV waves are also clearly correlated with several other dynamic
phenomena, though there are still efforts underway to distinguish
correlative vs. causal relationships.  These studies are extremely
important, as they provide clear clues as to the origin and physical
nature of the waves.  For example, \cite{2002ApJ...569.1009B}
demonstrated that CMEs show a much greater association to EUV waves
than do flares, while \cite{2006ApJ...641L.153} found that only
eruptive flares were associated with EUV waves.  Other investigations,
such as \cite{2000SoPh..193..161T} and \cite{2004A&A...427..705Z} and
\cite{2010ApJ...709..369P} have also indicated that the development of
coronal dimmings may be closely linked to the development of EUV
waves.

The path of EUV waves have been observed to be modified by nearly all
major coronal features, including active regions (Wang 2000),
filaments (Liu et al. 2012), coronal holes (e.g., Gopalswamy et
al. 2009), streamers (e.g., Kwon et al. 2013), and with varying
degrees of transmission, refraction, reflection and absorption reveal
details about the wave interaction with these features. However, the
impulsive excitation, and the range of interactions between the EUV
waves and other coronal structures are still unknown.

EUV waves can be used to infer properties of the coronal medium that
are otherwise hard to measure, i.e., they may be used as tools to
perform coronal seismology \citep{1970PASJ...22..341U}.  If one
assumes that an EUV wave propagates as a fast MHD wave mode (see
\cite{2011SSRv..158..365G}, for a review of current interpretations of
EUV waves), and the wave propagation speed, coronal density and
temperature can all be estimated, then the coronal magnetic field
strength can be derived \cite{2005LRSP....2....3N}.  This value can
also be used to test the accuracy of magnetic field extrapolation
codes \citep{2008ApJ...675.1637S} and other indirect measurements of
the coronal magnetic field strength \citep{2007Sci...317.1192T}.
